\documentclass[a4paper,12pt]{article}
\usepackage{hyperref}
\begin{document}
\title{Chomp
\subsubsection*{Requirements and Specification Document -- Group 6}\subsubsection*{Version 0.1}
}\maketitle
\section*{Project Abstract}
Chomp is a web application that tracks what a person, group, or family eats, and gives nutritional advice, financial suggestions, sustainability ratings, and other meal and dietary statistical analysis.  Chomp provides a more digestable understanding of how our food affects us beyond a simple calorie counter.  By leveraging multiple informational APIs, a wealth of data can be generated from simple meal entries.  Customers use Chomp to gague their daily, weekly, monthly, and yearly progress toward their personal goals, and choose Chomp as their preffered tool because of its variety and wealth of information packed into a convenient and understandable form.  Plans for Chomp include adding a social support network of like-minded users to aid in users' feeling of and drive for success.  The anticipated revenue stream for Chomp will be in-app addvertisements that target a given user.

Our github location is \url{http://github.com/hedekar/chomp}
\section*{Customers}
\subsubsection*{Weight Loss}
This customer uses Chomp to keep a daily count of their calories and specific nutritional information to assist them in losing weight.  See Appendix 1 for the user persona of Boris Chetznekov for further info.
\subsubsection*{Family Nutrition}
This customer uses Chomp to track how nutritious her family meals are and what deficiencies in the diet may need to be addressed.  She is also very concerned with meal cost.  See Appendix 1 for the user persona of Susan Montgomery for further info.
\subsubsection*{Body Builder}
This customer uses Chomp to monitor his body's nutrition levels as he prepares for body building competitions.  He lives with two roommates who also work out but not to the same level of competitiveness, but they all share groceries.  See Appendix 1 for the user persona of Chet Reist for further info.
\section*{Competitive Analysis}
Many other meal-tracking services already exist.  Some have multiple million downloads and users.  The majority of these however are very specific in scope, with only calories being counted or with a focus on reporting all the nutritional numbers at once.  The large majority of these applications are targeting weight loss customers, and not an everyday user.

Both web applications and mobile applications have been developed in this field.  Some specific competitors include: \begin{list}{•}{•}
\item \textbf{Noom Coach}
\item \textbf{MyFitnessPal}
\item \textbf{Lifesum}
\item \textbf{FatSecret}
\item \textbf{My Food Diary}
\end{list}
\section*{User Stories}
\subsection{Login}
\subsubsection{Registration}
A user is greeted with an interface requesting to register.  Options for both Google and Facebook account linkage (via their account APIs) will be presented as well as a manual entry field for e-mail and password.  The manual entry process will request first name, last name, username, and e-mail.  The registration section will send the user directly to the Account Setup stage.
\subsubsection{Account Setup}

\subsubsection{Login}

\subsubsection{Logout}

\section*{User Interface Requirements}

\newpage 
\section*{Appendix 1 - User Personas}
\subsection*{Boris Chetznekov}
Age: 65
\subsection*{Susan Montgomery}
Age: 32
\subsection*{Chet Reist}
Age: 21
\end{document}